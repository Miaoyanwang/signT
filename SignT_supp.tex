\documentclass[11pt]{article}

\usepackage{fancybox}



\usepackage{color}
\usepackage{url}
\usepackage[margin=1in]{geometry}


\renewcommand{\textfraction}{0.0}
\renewcommand{\topfraction}{1.0}
%\renewcommand{\textfloatsep}{5mm}


\usepackage{comment}
% Definitions of handy macros can go here
\usepackage{amsmath,amssymb,amsthm,bm,mathtools}
\usepackage{multirow}
\usepackage{natbib}
%\usepackage{dsfont,multirow,hyperref,setspace,natbib,enumerate}
\usepackage{dsfont,multirow,hyperref,setspace,enumerate}
\hypersetup{colorlinks,linkcolor={blue},citecolor={blue},urlcolor={red}}
\usepackage{algpseudocode,algorithm}
\algnewcommand\algorithmicinput{\textbf{Input:}}
\algnewcommand\algorithmicoutput{\textbf{Output:}}
\algnewcommand\INPUT{\item[\algorithmicinput]}
\algnewcommand\OUTPUT{\item[\algorithmicoutput]}

\mathtoolsset{showonlyrefs=true}



\theoremstyle{plain}
\newtheorem{thm}{Theorem}[section]
\newtheorem{lem}{Lemma}[section]
\newtheorem{prop}{Proposition}[section]
\newtheorem{pro}{Property}[section]
\newtheorem{assumption}{Assumption}[section]

\theoremstyle{definition}
\newtheorem{defn}{Definition}[section]
\newtheorem{cor}{Corollary}[section]
\newtheorem{example}{Example}[section]
\newtheorem{rmk}{Remark}[section]


\renewcommand{\thefigure}{{S\arabic{figure}}}%
\renewcommand{\thetable}{{S\arabic{table}}}%
\renewcommand{\figurename}{{Supplementary Figure}}
\renewcommand{\tablename}{{Supplementary Table}}
\setcounter{figure}{0}
\setcounter{table}{0}


\def\MLET{\hat \Theta_{\text{MLE}}}
\newcommand{\cmt}[1]{{\leavevmode\color{red}{#1}}}



\usepackage{dsfont}

\usepackage{multirow}

\DeclareMathOperator*{\minimize}{minimize}

\usepackage{wrapfig}

\usepackage{mathtools}
\usepackage{enumitem}
\mathtoolsset{showonlyrefs}
\newcommand*{\KeepStyleUnderBrace}[1]{%f
  \mathop{%
    \mathchoice
    {\underbrace{\displaystyle#1}}%
    {\underbrace{\textstyle#1}}%
    {\underbrace{\scriptstyle#1}}%
    {\underbrace{\scriptscriptstyle#1}}%
  }\limits
}
\usepackage{xr}
\externaldocument{signT}
\input macros.tex


\def\sign{\textup{sgn}}
\def\srank{\textup{srank}}
\def\rank{\textup{rank}}
\def\caliP{\mathscr{P}_{\textup{sgn}}}
\def\risk{\textup{Risk}}



\title{Supplements for ``Beyond the Signs: Nonparametric Tensor Completion via Sign Series''}


\begin{document}

\begin{center}
\begin{spacing}{1.5}
\textbf{\Large Appendix for ``Beyond the Signs: Nonparametric Tensor Completion via Sign Series''}
\end{spacing}
\end{center}

\appendix
The appendix consists of additional theoretical results (Section~\ref{sec:additional}), numerical experiments (Section~\ref{sec:data}), and proofs (Section~\ref{sec:proof}). 

\section{Additional results}\label{sec:additional}
\subsection{Sensitivity of tensor rank to monotonic transformation}
In Section~\ref{sec:intro} of the main paper, we have provided a motivating example to show the sensitivity of tensor rank to monotonic transformations. Here, we describe the details of the example set-up. 

The step 1 is to generate a rank-3 tensor $\tZ$ based on the CP representation
\[
\tZ=\ma^{\otimes 3}+\mb^{\otimes 3}+\mc^{\otimes 3},
\]
where $\ma,\mb,\mc\in\mathbb{R}^{30}$ are vectors consisting of $N(0,1)$ entries. Then we apply $f(z)=(1+\exp(-cz))^{-1}$ to $\tZ$ entrywise, and obtain a transformed tensor $\Theta=f(\tZ)$. 

The step 2 is to determine the rank of $\Theta$. Unlike matrices, the exact rank determination for tensors is NP hard. Therefore, we choose to compute the numerical rank of $\Theta$ as an approximation.  The numerical rank of $\Theta$ is determined as the minimal rank for which the relative approximation error is below $0.1$, i.e.,
\begin{equation}\label{eq:numeric}
 \hat r(\Theta)=\min\left\{s\in\mathbb{N}_{+}\colon \min_{\hat \Theta\colon \rank(\hat \Theta)\leq s}{\FnormSize{}{\Theta-\hat \Theta}\over \FnormSize{}{\Theta}} \leq 0.1\right\}.
\end{equation}
We compute $\hat r(\Theta)$ by searching over $s\in\{1,\ldots,30^2\}$, where for each $s$, we (approximately) solve the least-square minimization using CP function in R package {\tt rTensor}. 

We repeat steps 1-2 ten times, and report the averaged numerical rank of $\Theta$ along with standard errors in Figure~\ref{fig:example}a.  

\subsection{Tensor rank and sign-rank}\label{sec:high-rank}
In the main paper, we have provided several tensor examples with high usual rank but low sign-rank. This section provides more examples and their proofs. 

Unless otherwise specified, let $\Theta$ be an order-$K$ $(d,\ldots,d)$-dimensional tensor. 
\begin{example}[Max hypergraphon]\label{example:max} Suppose the tensor $\Theta$ takes the form 
\[
\Theta(i_1,\ldots,i_K)=\log\left(1+{1\over d}\max(i_1,\ldots,i_K)\right), \ \text{for all }(i_1,\ldots,i_K)\in[d]^K.
\]
 Then 
 \[
 \rank(\Theta)\geq d, \quad \text{and}\quad \srank(\Theta-\pi)\leq 2\ \text{for all }\pi\in\mathbb{R}. 
 \]
\end{example}
\begin{proof}
We first prove the results for $K=2$. 

Symbolic calculation in {\tt Mathematica} shows that $\rank(\Theta)=d$ for $K=2$ \fixme{Miaoyan}{Verify by Mathematica or by hand}. It suffices to show $\srank(\Theta-\pi)\leq 2$ for $\pi$ in the feasible range $\pi \in (\log(1+{1\over d}),\ \log 2)$. When $\pi$ is in this range, there exists an index $i^*\in\{2,\ldots,d\}$, such that $\log(1+{i^*-1\over d})< \pi\leq \log(1+{i^*\over d})$. By definition, the sign matrix $\sign (\Theta-\pi)$ takes the form
\begin{equation}\label{eq:matrix}
\sign (\Theta(i,j)-\pi)=
\begin{cases}
-1, & \text{both $i$ and $j$ are smaller than $i^*$};\\
1, & \text{otherwise}.
\end{cases}
\end{equation}
Therefore, the matrix $\sign (\Theta-\pi)$ is a rank-2 block matrix. By definition of sign rank, we have $\srank(\Theta-\pi)=2$. 

We now extend the results to $K\geq 3$. Based on the definition, the rank of a tensor is lower bounded by the rank of its matrix slice. It follows that $\rank(\Theta)\geq \rank(\Theta(\colon,\colon,1,\ldots,1))=d$. For the sign rank with feasible $\pi$, notice that the sign tensor $\sign(\Theta-\pi)$ takes the similar form as in~\eqref{eq:matrix},
\begin{equation}\label{eq:entrywise}
\sign (\Theta(i_1,\ldots,i_K)-\pi)=
\begin{cases}
-1, & \text{$i_k<i^*$ for all $k\in[K]$};\\
1, & \text{otherwise}.
\end{cases}
\end{equation}
The identity~\eqref{eq:entrywise} implies that $\sign(\Theta-\pi)=-2\ma^{\otimes K}+1$, where $\ma=(1,\ldots,1,0,\ldots,0)^T$ takes 1 on $i$-th entry for $i<i^*$ and 0 otherwise. Henceforth $\srank(\Theta-\pi)=2$. 
\end{proof}

In fact, the example~\ref{example:max} is a special case of the following Proposition. 

\begin{prop}[Min/Max hypergraphon] Let $\tZ_{\max}\in\mathbb{R}^{d_1\times \cdots \times d_K}$ denote a tensor with entries 
\begin{equation}\label{eq:max}
\tZ_{\max}(i_1,\ldots,i_K)=\max(x^{(1)}_{i_1},\ldots,x^{(K)}_{i_K}),
\end{equation}
where $x^{(1)}_{i_1}, \ldots, x^{(K)}_{i_K} \in[0,1]$ are pre-specified numbers for all $i_k\in[d_k]$ and $k\in[K]$. Let $g\colon \mathbb{R}\to \mathbb{R}$ be a continuous function and $\Theta:=g(\tZ_{\max})$ be the transformed tensor. Fix $\pi\in\mathbb{R}$. Suppose the function $g(z)=\pi$ has at most $r\geq 1$ distinct real roots, then sign rank of $(\Theta-\pi)$ satisfies
\[
\srank(\Theta-\pi)\leq 2r.
\]
The same conclusion holds if we use $\min$ in place of $\max$ in~\eqref{eq:max}. 
\end{prop}
\begin{proof} 
We reorder the tensor index along each mode such that $x^{(k)}_{1}\leq \cdots \leq x^{(k)}_{d_k}$ for all $k\in[K]$. Based on the construction of $\tZ_{\max}$, the reordering does not change the rank of $\tZ_{\max}$ or $(\Theta-\pi)$. Let $z_1<\cdots<z_r$ be the $r$ distinct real roots for the equation $g(z)=\pi$. We separate the proof for two cases, $r=1$ and $r\geq 2$. 

\begin{itemize}[leftmargin=*,topsep=0pt,itemsep=-1ex,partopsep=1ex,parsep=1ex]
\item When $r=1$. The continuity of $g(\cdot)$ implies that the function $(g(z)-\pi)$ has at most one sign change point. Using similar proof as in Example~\ref{example:max}, we have
\[
\sign(\Theta-\pi)=1-2\ma^{(1)}\otimes\cdots\otimes \ma^{(K)},\quad \text{where}\quad \ma^{(k)}=(\KeepStyleUnderBrace{1,\ldots,1,}_{\text{positions for which $x_{i_k}^{k}<z_1$}}0,\ldots,0)^T \text{ for $k\in[K].$}
\]
Therefore, $\srank(\Theta-\pi)\leq \rank(\sign(\Theta-\pi)) = 2$. 

\item When $r\geq 2$.  Based on the continuity, the function$(g(z)-\pi)$ is non-zero and remains an unchanged sign in each of the intervals $(z_s, z_{s+1})$ for $1\leq s\leq r-1$. Define the index set $\tI=\{s\in\mathbb{N}_{+}\colon \text{the interval $(z_s, z_{s+1})$ in which $g(z)<\pi$}\}$. 
We now prove that the sign tensor $\sign(\Theta-\pi)$ has rank bounded by $2r-1$. To see this, consider the tensor indices for which $\sign(\Theta-\pi)=-1$,
\begin{align}\label{eq:support}
\{\omega\colon \Theta(\omega)-\pi <0 \} & = \{\omega \colon g(\tZ_{\max}(\omega))<\pi\} \notag \\
&=\cup_{s\in \tI} \{\omega\colon \tZ_{\max}(\omega)\in(z_s,z_{s+1})\}\notag\\
&=\cup_{s\in \tI} \{\omega\colon \text{$x^{(k)}_{i_k}< z_{s+1}$ for all $k\in[K]$}\}\cap \{\omega\colon \text{$x^{(k)}_{i_k}\leq z_{s}$ for all $k\in[K]$}\}^c.
\end{align}
The identity~\eqref{eq:support} is equivalent to 
\begin{align}\label{eq:indicator}
\mathds{1}(\Theta(i_1,\ldots,i_K)< \pi)&
=\sum_{s\in \tI}\left( \prod_k \mathds{1}(x^{(k)}_{i_k}< z_{s+1}) - \prod_k \mathds{1}(x^{(k)}_{i_k}\leq z_{s})\right),\ 
%=\sum_{s\in \tI}\gamma_s \left[1-\mathds{1}(x_{i_1}\notin(z_s,z_{s+1}))\times \cdots \times \mathds{1}(x_{i_K}\notin(z_s,z_{s+1}))\right]
%&=\sum_{s\leq r-1}\gamma_s\ma^{(1)}_s\otimes\cdots\otimes \ma^{(K)}_s
\end{align}
for all $(i_1,\ldots,i_K)\in[d_1]\times \cdots\times[d_K]$, where $\mathds{1}(\cdot)\in\{0,1\}$ denotes the indicator function. The equation~\eqref{eq:indicator} implies the low-rank representation of $\sign(\Theta-\pi)$,
\begin{equation}\label{eq:sum}
\sign(\Theta-\pi)=1-2\sum_{s\in \tI } \left(\ma^{(1)}_{s+1}\otimes\cdots\otimes \ma^{(K)}_{s+1} - \bar \ma^{(1)}_s\otimes\cdots\otimes \bar \ma^{(K)}_s\right),
\end{equation}
where we have denoted the two binary vectors 
\[
\ma^{(k)}_{s+1}=(\KeepStyleUnderBrace{1,\ldots,1,}_{\text{positions for which $x_{i_k}^{(k)}<z_{s+1}$}}0,\ldots 0)^T,\quad \text{and}\quad
\bar \ma^{(k)}_s=(\KeepStyleUnderBrace{1,\ldots,1,}_{\text{positions for which $x_{i_k}^{(k)}\leq z_{s}$}}0,\ldots 0)^T.
\]
Note that $|\tI|\leq r-1$. Therefore we conclude that 
\[
\srank(\Theta-\pi)\leq 1+2(r-1)=2r-1.
\]
\end{itemize}
Combining two cases yield that $\srank(\Theta-\pi)\leq 2r$ for any $r\geq 1$.
\end{proof}

We provide several additional examples such that $\rank(\Theta)\geq d$ whereas $\srank(\Theta)\leq c$ for a constant $c$ independent of $d$. We state the examples in the matrix case, i.e, $K=2$. Similar conclusion extends to $K\geq 3$, because of the following Proposition. 
\begin{prop} Let $\mM\in\mathbb{R}^{d_1\times d_2}$ be a matrix. For any given $K\geq 3$, define an order-$K$ tensor $\Theta\in\mathbb{R}^{d_1\times \cdots \times d_K}$ by
\[
\Theta=\mM\otimes \mathbf{1}_{d_3}\otimes \cdots \otimes \mathbf{1}_{d_K},
\] 
where $\mathbf{1}_{d_k}\in\mathbb{R}^{d_k}$ denotes an all-one vector, for $3\leq k\leq K$. Then we have
\[
\rank(\Theta)=\rank(\mM),\quad \text{and}\quad \srank(\Theta-\pi)=\srank(\mM-\pi) \ \text{for all $\pi\in\mathbb{R}$}.
\] 
\end{prop}
\begin{proof}
The conclusion directly follows from the definition of tensor rank. 
\end{proof}

\begin{example}[Stacked banded matrices]\label{example:banded} Let $\ma=(1,2,\ldots,d)^T$ be a $d$-dimensional vector, and define a $d$-by-$d$ banded matrix $\mM=|\ma\otimes \mathbf{1}-\mathbf{1}\otimes \ma|$. Then
\[
\rank(\mM)=d,\quad \text{and}\quad \srank(\mM-\pi)\leq 3, \quad \text{for all }\pi\in \mathbb{R}.
\]
\end{example}
\begin{proof}
Note that $\mM$ is a banded matrix with entries
\[
\mM(i,j)={|i-j|}, \quad \text{for all }(i,j)\in[d]^2.
\]
Straightforward calculation shows that $\mM$ is full rank. \fixme{Miaoyan}{Verify.}

We now show $\srank(\mM-\pi)\leq 3$ by construction. Define two vectors $\mb=(2^{-1},2^{-2},\ldots,2^{-d})^T\in\mathbb{R}^d$ and $\text{rev}(\mb)=(2^{-d},\ldots,2^{-1})^T\in\mathbb{R}^d$. We construct the following matrix
\begin{equation}\label{eq:A}
\mA=\mb\otimes\mb+\text{rev}(\mb)\otimes \text{rev}(\mb).
\end{equation}
It is easy to see that $\mA\in\mathbb{R}^{d\times d}$ is a banded matrix with entries
\[
\mA(i,j)=\mA(j,i)=\mA(d-i,d-j)=\mA(d-j,d-i)=2^{-d-1}\left(2^{j-i}+2^{i-j}\right),\ \text{for all }(i,j)\in[d]^2.
\] 
Furthermore, the entry value $\mA(i,j)$ decreases with respect to $|i-j|$, i.e., 
\begin{equation}\label{eq:decrease}
\mA(i,j) \geq \mA(i',j'), \quad \text{for all }|i-j|\geq |i'-j'|.
\end{equation}
Now, for a given $\pi\in\mathbb{R}$, there exists $\pi'\in\mathbb{R}$ such that $\sign(\mA-\pi')=\sign(\mM-\pi)$. This is because both $\mA$ and $\mM$ are banded matrices satisfying monotonicity~\eqref{eq:decrease}. By definition~\eqref{eq:A}, $\mA$ is a rank-2 matrix. Henceforce, $\srank(\mM-\pi)=\srank(\mA-\pi')\leq 3.$
\end{proof}

\begin{rmk} The tensor analogy of banded matrices $\Theta=|\ma\otimes\mathbf{1}\otimes \mathbf{1}-\mathbf{1}\otimes\ma\otimes \mathbf{1}|$ is used as simulation model 3 in the main paper.  
\end{rmk}

\begin{example}[Stacked identity matrices]
Let $\mI$ be a $d$-by-$d$ diagonal matrix. Then
\[
\rank(\mI)=d,\quad\text{and}\quad  \srank(\mI-\pi)\leq 3 \ \text{for all }\pi\in\mathbb{R}.
\]
\end{example}
\begin{proof}
Depending on the value of $\pi$, the sign matrix $\sign(\mI-\pi)$ fall into one of the three cases: 1) $\sign(\mI-\pi)$ is a matrix of all 1; 2) $\sign(\mI-\pi)$ is a matrix of all -1; 3) $\sign(\mI-\pi)=\sign(\mI)$. The former two cases are trivial, so it suffices to show $\srank(\mI)\leq 3$. 


Based on Example~\ref{example:banded}, the rank-2 matrix $\mA$ in~\eqref{eq:A} satisfies 
\[
\mA(i,j)
\begin{cases}
=2, & i=j,\\
\geq 2+{1\over 2}, & i\neq j.
\end{cases}
\]
Therefore $\sign(2+{1\over 4}-\mA)=\sign(\mI)$. We conclude that $\srank(\mI)\leq \rank(2+{1\over 4}-\mA)=3$. 


\end{proof}


\section{Additional data analysis}~\label{sec:data}
\subsection{Brain connectivity analysis}
Table~\ref{tab:??} shows the MAE based on cross-validation with $r = 3,6,9,12,15$ and $H = 20$. We find that our method outperforms substantially in all combinations of ranks and missing rates. In addition, standard error of MAE from five repeated cross-validation shows that our method is more stable than CPT. One advantage of our method is that our estimation is always bounded in $[0,1]$ whereas CPT estimation may fall outside the valid range $[0,1]$. 
\fixme{Miaoyan}{Insert the table (or figure). explain briefly the regression analysis}


\subsection{NIPS data analysis}
In the main paper we have summarized the MAE from cross validation for $r=6, 9,12$. Here we provide additional results for a wider range $r= 3, 6, 9,12,15$. Table~\ref{tab:NIPS} suggests that further increment of rank appears to have little effect on the performance. In addition, we also perform naive imputation where the missing values are predicted using the observed sample average. The two tensor methods outperform the naive imputation, implying the necessity of incorporating tensor structure in the analysis.
\begin{table}[H]
\centering
\begin{tabular}{c|ccccc}
Method & $r = 3$ & $r = 6$ & $r=9$ & $r=12$&$r=15$ \\
\hline
NonparaT (Ours) & {\bf 0.18}(0.002) & {\bf 0.16}(0.002) & {\bf 0.15}(0.001)& {\bf 0.14}(0.001)&{\bf 0.13}(0.001)\\
 \hline
Low-rank CPT &0.22(0.004) & 0.20(0.007) & 0.19(0.007)&0.17(0.007)&0.17(0.007)\\
  \hline
Naive imputation (Baseline)& \multicolumn{5}{c}{0.32(.001)}\\
\end{tabular}
\caption{Prediction accuracy measured in MAE in the NIPS data analysis. The reported MAEs are averaged over five runs of cross-validation, with standard errors in parentheses. Bold numbers indicate the minimal MAE among three methods. For low-rank CPT, we use R function {\tt rTensor} with default hyperparameters, and for our method, we set $H=20$. }\label{tab:NIPS}
\end{table}


\section{Proofs}\label{sec:proof}
\subsection{Proofs of Propositions~\ref{cor:monotonic}-\ref{prop:global}}
\begin{proof}[Proof of Proposition~\ref{cor:monotonic}]
Based on the definition of sign rank, it suffices to show that $\Theta\simeq g(\Theta)$. The strictly monotonicity of $g$ implies that the inverse function $g^{-1}\colon \mathbb{R}\to \mathbb{R}$ is well-defined. 

Furthermore, the mapping $x\mapsto g(x)$ is sign preserving. Specifically, if $x\geq 0$, then $g(x)\geq g(0)=0$. On the other hand, if $g(x)\geq 0=g(0)$, then applying $g^{-1}$ to both sides gives $x\geq 0$. Therefore, $\Theta\simeq g(\Theta)$, and $\srank(\Theta)=\srank(g(\Theta))\leq \rank (g(\Theta))$. 
\end{proof}


\begin{proof}[Proof of Proposition~\ref{cor:broadness}]
See Section~\ref{sec:high-rank} for constructive examples.
\end{proof}

\begin{proof}[Proof of Proposition~\ref{prop:global}]
Based on the definition, the function $\risk(\cdot)$ relies only on the sign pattern of the tensor. Therefore, without loss of generality, we assume both tensors $\bar \Theta, \tZ \in\{-1,1\}^{d_1\times \cdots \times d_K}$ are binary tensors. 
We evaluate the excess risk 
\begin{equation}\label{eq:risk}
\risk(\tZ)- \risk(\bar \Theta) = \mathbb{E}_{\omega\sim \Pi}\KeepStyleUnderBrace{\mathbb{E}_{\tY(\omega)}\left\{|\tY(\omega)-\pi|\left[\left|\tZ(\omega)-\sign(\bar \tY(\omega)) \right|-\left|\bar\Theta(\omega)-\sign(\bar \tY(\omega))\right|\right]\right\}}_{=:I(\omega)}.
\end{equation}
Denote $y=\tY(\omega)$, $z=\tZ(\omega)$, $\bar \theta=\bar\Theta(\omega)$, and $\theta=\Theta(\omega)$. It follows from the expression of $I(\omega)$ that
\begin{align}\label{eq:I}
I(\omega)&= \mathbb{E}_{y}\left[ (y-\pi)(\bar \theta-z)\mathds{1}(y\geq \pi)+(\pi-y)(z-\bar \theta)\mathds{1}(y< \pi)\right]\notag \\
&= \mathbb{E}_{y}\left[(\bar \theta-z) (y-\pi)\right]\notag \\
&=  \left[\sign(\theta-\pi)-z\right]\left(\theta-\pi\right)\notag \\
&= |\sign(\theta-\pi)-z||\theta-\pi|\geq 0,
\end{align}
where the third line uses the fact $\mathbb{E}y=\theta$ and $\bar \theta=\sign(\theta-\pi)$, and the last line uses the assumption $z \in\{-1,1\}$. The equality~\eqref{eq:I} is attained when $z=\sign(\theta-\pi)$ or $\theta=\pi$. Combining~\eqref{eq:I} with~\eqref{eq:risk}, we conclude that, for all $\tZ\in\{-1,1\}^{d_1\times \cdots \times d_K}$, 
\begin{equation}\label{eq:minimum}
\risk(\tZ)- \risk(\bar \Theta) = \mathbb{E}_{\omega\sim \Pi} |\sign(\Theta(\omega)-\pi)-\tZ(\omega)||\Theta(\omega)-\pi|\geq 0,
\end{equation}
In particular, setting $\tZ=\bar \Theta=\sign(\Theta-\pi)$ in~\eqref{eq:minimum} yields the minimum. Therefore, 
\[
\risk(\bar \Theta)=\min\{\risk(\tZ)\colon \tZ\in \mathbb{R}^{d_1\times \cdots \times d_K}\} \leq \min\{\risk(\tZ)\colon \rank(\tZ)\leq r\}.
\]
Because $\srank(\Theta-\pi)\leq r$ by assumption, the last inequality becomes equality. The proof is complete. 
\end{proof}

\subsection{Proof of Theorem~\ref{thm:population}}
\begin{proof}[Proof of Theorem~\ref{thm:population}]
Based on~\eqref{eq:minimum} in Proposition~\ref{prop:global} we have
\begin{equation}\label{eq:population2}
\risk(\tZ)- \risk(\bar \Theta) = \mathbb{E}\left[|\sign \tZ-\sign\bar \Theta||\bar \Theta|\right].
\end{equation}
The Assumption~\ref{ass:margin} states that
\begin{equation}\label{eq:ass}
\mathbb{P}\left(|\bar \Theta | \leq t\right)\leq ct^\alpha,\quad \text{for all } 0\leq t< \rho(\pi,\tN).
\end{equation}
Without future specification, all relevant probability statements, such as $\mathbb{E}$ and $\mathbb{P}$, are with respect to $\omega\sim \Pi$. 

We divide the proof into two cases: $\alpha >0$ and $\alpha = 0$.
\begin{itemize}[leftmargin=*,topsep=0pt,itemsep=-1ex,partopsep=1ex,parsep=1ex]
\item Case 1: $\alpha>0$. 

Let $t$ be an arbitrary number in the interval $[0,1]$, and define the index set $E=\{\omega \colon |\bar \Theta(\omega)|>t\}$. By~\eqref{eq:population2}, for all $0\leq t< \rho(\pi, \tN)$,
\begin{align}\label{eq:1}
\risk(\tZ)- \risk(\bar \Theta) &\geq t\mathbb{P}\left(\sign\tZ \neq \sign \bar \Theta\text{ and }|\bar \Theta|>t   \right)\notag \\
& \geq t\Big\{\mathbb{P}\left(\sign\tZ \neq \sign \bar \Theta \right) - \mathbb{P}\left(|\bar \Theta|>t\right)\Big\}\notag\\
&\geq {t\over 2}\Big\{\textup{MAE}(\sign \tZ, \sign \bar \Theta) - 2ct^\alpha \Big\},
\end{align}
where the last line follows from the definition of MAE and~\eqref{eq:ass}. We maximize the lower bound~\eqref{eq:1} with respect to $t$, and obtain the optimal $t_{\text{opt}}$,
\[
t_{\text{opt}}=\begin{cases}
\rho(\pi, \tN), & \text{if } \textup{MAE}(\sign \tZ,\sign \bar\Theta) \geq 2c(1+\alpha) \rho^{\alpha}(\pi, \tN),\\
\left[ {\alpha \over 2c(1+\alpha)} \textup{MAE} (\sign \tZ,\sign \bar\Theta)  \right]^\alpha, &  \text{if }\textup{MAE}( \sign \tZ,\sign \bar\Theta) \leq 2c(1+\alpha) \rho^{\alpha}(\pi, \tN),
 \end{cases}
\]
The corresponding lower bound of the inequality~\eqref{eq:1} becomes
\[
\risk(\tZ)- \risk(\bar \Theta) \geq 
\begin{cases}
c_1 \rho(\pi, \tN) \textup{MAE}(\sign \tZ,\sign \bar\Theta),  & \text{if } \textup{MAE}(\sign \tZ,\sign \bar\Theta) \geq 2c(1+\alpha) \rho^{\alpha}(\pi, \tN),\\
c_2 \left[ \textup{MAE}( \sign \tZ,\sign \bar\Theta)\right]^{1+\alpha \over \alpha}, & \text{if }\textup{MAE}(\sign \tZ,\sign \bar\Theta) \leq 2c(1+\alpha) \rho^{\alpha}(\pi, \tN),
\end{cases}
\]
where $c_1,c_2>0$ are two constants independent of $\tZ$. Combining both cases gives
\begin{align}\label{eq:MAE}
\textup{MAE}(\sign \tZ,\sign \bar\Theta) & \lesssim [\risk(\tZ)- \risk(\bar \Theta)]^{\alpha\over 1+\alpha}+{1\over \rho(\pi, \tN)} \left[\risk(\tZ)- \risk(\bar \Theta)\right]\\
&\leq C(\pi)[\risk(\tZ)- \risk(\bar \Theta)]^{\alpha\over 1+\alpha},
\end{align}
where $C(\pi)>0$ is a multiplicative factor independent of $\tZ$. 
\item Case 2: $\alpha=0$. The inequality~\eqref{eq:1} now becomes
\begin{equation}\label{eq:2}
\risk(\tZ)- \risk(\bar \Theta) \geq 2t\textup{MAE}(\sign \bar\Theta, \sign \tZ), \quad \text{for all }0\leq t< \rho(\pi,\tN).
\end{equation}
The conclusion follows by taking $t={\rho(\pi, \tN)\over 2}$ in the inequality~\eqref{eq:2}. 
\end{itemize}
\end{proof}
\begin{rmk}\label{eq:rmk}The proof of Theorem~\ref{thm:population} shows that, under Assumption~\ref{ass:margin}, 
\begin{equation}\label{eq:remark}
\textup{MAE}(\sign \tZ,\sign \bar \Theta)  \lesssim [\risk(\tZ)- \risk(\bar \Theta)]^{\alpha\over 1+\alpha}+{1\over \rho(\pi, \tN)} \left[\risk(\tZ)- \risk(\bar \Theta)\right],
\end{equation}
for all $\tZ\in\mathbb{R}^{d_1\times \cdots \times d_R}$. For fixed $\pi$, the second term is absorbed into the first term. 
\end{rmk}

\subsection{Proof of Theorem~\ref{thm:classification}}
\fixme{Miaoyan}{
Pay attention to 1. $K$. express the dependence on $K$ explicitly.
2. $\lesssim$ vs $\leq$. 3. Notational consistence with main paper. 4. selfcontainness of each lemma, thm.}
The following lemma shows variance-to-mean relationship implied by the $\alpha$-smoothness of $\Theta$. The relationship plays a key role in determining the convergence rate based on empirical process theory~\citep{shen1994convergence}. 
\begin{lem}[Variance-to-mean relationship]\label{lem:variance}
Consider the same setup as in Theorem~\ref{thm:classification}. Fix $\pi\in[-1,1]$. Denote the $\pi$-shifted signal $\bar \Theta=\Theta-\pi$, the $\pi$-shifted data $\bar \tY=\tY-\pi$, the $\pi$-weighted classification objective function
\begin{equation}\label{eq:sample}
L(\tZ, \bar \tY_\Omega)= {1\over |\Omega|}\sum_{\omega \in \Omega}\ \KeepStyleUnderBrace{|\bar \tY(\omega)|}_{\text{weight}}\  \times \ \KeepStyleUnderBrace{| \sign \tZ(\omega)-\sign \bar \tY(\omega)|}_{\text{classification loss}},
\end{equation}
and $\risk(\tZ)=\mathbb{E}L(\tZ,\bar \tY_\Omega)$.  Under Assumption~\ref{ass:margin} of the $(\alpha,\pi)$-smoothness of $\Theta$, we have
\begin{equation}\label{eq:variance}
\textup{Var}[L(\tZ,\bar \tY_\Omega)-L(\bar \Theta, \bar \tY_\Omega)]\leq [\textup{Risk}(\tZ)-\textup{Risk}(\bar \Theta)]^{\alpha \over 1+\alpha}+{1\over \rho(\pi, \tN)}[\textup{Risk}(\tZ)-\textup{Risk}(\bar \Theta)],
\end{equation}
for all tensors $\tZ\in\mathbb{R}^{d_1\times \cdots \times d_K}$. Here the expectation and variance are taken with respect to both $\tY$ and $\omega\sim \Pi$. 
\end{lem}
\begin{proof}
We expand the variance by
\begin{align}\label{eq:mae}
\text{Var}[L(\tZ,\bar \tY_\Omega)-L(\bar \Theta, \bar \tY_\Omega)] &\leq \mathbb{E}|L(\tZ,\bar \tY_\Omega)-L(\bar \Theta, \bar \tY_\Omega)|^2\notag \\
&\lesssim \mathbb{E}|L(\tZ,\bar \tY_\Omega)-L(\bar \Theta, \bar \tY_\Omega)|\notag \\
&\leq \mathbb{E}|\sign\tZ-\sign \bar \Theta| = \textup{MAE}(\sign\tZ, \sign \bar \Theta),
\end{align}
where the second line comes from the boundedness of classification loss $L(\cdot ,\cdot)$, and the last line comes from the inequality $||z-y|-|\theta-y||\leq |z-\theta|$ for $z,y,\theta\in\{-1,1\}$. The conclusion~\eqref{eq:variance} then directly follows by applying remark~\ref{eq:rmk} to~\eqref{eq:mae}.
\end{proof}

\begin{proof}[Proof of Theorem~\ref{thm:classification}]
Fix $\pi\in[-1,1]$. For notational simplicity, we suppress all subscripts related to $\pi$, and write $\hat \tZ$ in place of $\hat \tZ_\pi$. Denote $n=|\Omega|$ and $\rho=\rho(\pi, \tN)$. 

Because the classification loss $L(\cdot, \cdot)$ is scale-free, i.e., $L(\tZ,\cdot)=L(c\tZ, \cdot)$ for every $c>0$, we consider the estimation subject to $\FnormSize{}{\tZ}\leq 1$ without loss of generality. Specifically, 
\[
\hat \tZ=\argmin_{\tZ\colon \textup{rank}(\tZ)\leq r, \FnormSize{}{\tZ}\leq 1}L(\tZ, \bar \tY_{\Omega}).
\]
We view $\tY_\Omega=\{\bar \tY(\omega)\colon \omega\in \Omega\}$ as a collection of $n$ i.i.d.\ random variables where the randomness is induced form both $\bar \tY$ and $\omega\sim\Pi$, and view the tensor $\tZ$ as a function that maps $\bar \tY_\Omega$ to $L(\tZ,\bar \tY_{\Omega})$. Then 
\[
\Delta_n(\tZ, \tY)
%={1\over n}\sum_{\omega\in \Omega} \KeepStyleUnderBrace{|\bar y_i||\sign z_i-\sign \bar y_i|}_{=:f(y_i)} 
:=L(\tZ, \bar \tY_{\Omega})-L(\bar \Theta, \bar \tY_{\Omega}),\quad \text{where }n=|\Omega|,\\ 
\]
is an empirical process induced by tensor $\tZ\in\tF=\{\tZ\colon \rank(\tZ)\leq r, \ \FnormSize{}{\tZ}\leq 1\}. $

Our remaining proof adopts the techniques of~\citet[Theorem 3]{wang2008probability} to the contexts of function family $\tZ\in \tF$. We summarize only the key difference here but refer to~\cite{wang2008probability} for complete proof. 
Based on Lemma~\ref{lem:variance}, the $(\alpha,\pi)$-smoothness of $\Theta$ implies 
%\[
%\textup{Var}[L(\tZ,\bar \tY_\Omega)-L(\bar \Theta, \bar \tY_\Omega)]\leq [\textup{Risk}(\tZ)-\textup{Risk}(\bar \Theta)]^{\alpha \over 1+\alpha}+{1\over \rho(\pi, \tN)}[\textup{Risk}(\tZ)-\textup{Risk}(\bar \Theta)].
%\]
\[
\textup{Var}\Delta_n(\tZ,\tY) \leq \left[\mathbb{E}\Delta_n(\tZ,\tY)\right]^{\alpha \over 1+\alpha}+{1\over \rho}\mathbb{E}\Delta_n(\tZ,\tY),\quad \text{for all $\tZ\in \tF$}.
\]
Applying the second order condition to Theorem 3 of \cite{wang2008probability} gives that
\begin{equation}\label{eq:rate}
\mathbb{P}\left(\risk(\hat \tZ)-\risk(\bar \Theta )\geq L_n\right)\leq C\exp(-nL^2_n),
\end{equation}
where the convergence rate $L_n>0$ is determined by the solution to the following inequality,
\begin{equation}\label{eq:equation}
{1\over L_n}\int_{L_n}^{\sqrt{L_n^{\alpha/(\alpha+1)}+{L_n\over \rho}}}\sqrt{\tH(\varepsilon,\tF, \vnormSize{}{\cdot}) }d\varepsilon \leq \sqrt{n}. 
\end{equation}
In particular, the smallest $L_n$ satisfying~\eqref{eq:equation} yields the best upper bound of the error rate. Here $\tH(\varepsilon, \tF,\vnormSize{}{\cdot})$ denotes the $L_2$-metric, $\varepsilon$-bracketing number (c.f. Definition~\ref{pro:inftynorm}) of family $\tF$. 

It remains to solve for the smallest possible $L_n$ in~\eqref{eq:equation}. Based on Lemma~\ref{lem:metric}, the inequality~\eqref{eq:equation} is satisfied with 
\begin{equation}\label{eq:tn}
L_n\asymp t_n^{(\alpha+1)/ (\alpha+2)} +{1\over \rho} t_n, \quad \text{where }t_n={Kd_{\max} r \over n}.
\end{equation}
Therefore, by~\eqref{eq:rate}, with very high probability. 
\[
\risk(\hat \tZ)-\risk(\bar \Theta )\leq t_n^{(\alpha+1)/(\alpha+2)} +{1\over \rho} t_n.
\]
Inserting the above bound into~\eqref{eq:remark} gives
\begin{align}
\textup{MAE}(\sign \hat \tZ, \sign \bar \Theta) &\lesssim [\risk(\hat \tZ)-\risk(\bar \Theta)]^{\alpha/(\alpha+1)}+{1\over \rho}[\risk(\hat \tZ)-\risk(\bar \Theta)]\\
&\lesssim t_n^{\alpha/(\alpha+2)}+{1\over \rho^{\alpha/\alpha+1}}t_n^{\alpha/(\alpha+1)}+{1\over \rho}t_n^{(\alpha+1)/(\alpha+2)}+{1\over \rho^2}t_n\\
&\leq 4t_n^{\alpha/(\alpha+2)}+{4\over \rho^2}t_n.
\end{align}
where the last line follows from the fact that $a(b^2+b^{2-\alpha}+b+1) \leq 4 a (b^2+1)$ with $a:={t_n \over \rho^2}$ and $b:=\rho n^{-1/(\alpha+2)}$. Plugging the form of $t_n$ in~\eqref{eq:tn} completes the proof. 
\end{proof}

\begin{defn}[Bracketing number]\label{pro:inftynorm}
Consider a set of functions $\tF$, and let $\varepsilon>0$. Let $\tX $ denote the domain space equipped with measure $\Pi$. We call $\{(f^l_m,f^u_m)\}_{m=1}^M$ an $L_2$-metric, $\varepsilon$-bracketing function set of $\tF$, if for every $f\in \tF$, there exists an $m\in[M]$ such that 
\[
f^l_m(x)\leq f(x)\leq f^u_m(x),\quad \text{for all }x\in\tX,
\]
and
\[
\vnormSize{}{f^l_m-f^u_m}\stackrel{\text{def}}{=}\sqrt{\mathbb{E}_{x\sim \Pi}|f^l_m(x)-f^u_m(x)|^2} \leq \varepsilon, \ \text{for all } m=1,\ldots,M. 
\]
The bracketing number with $L_2$-metric, $\tH_{[\ ]}(\varepsilon, \tF, \vnormSize{}{\cdot})$, is defined as the logarithm of the smallest cardinality of the $\varepsilon$-bracketing function set of $\tF$.  
\end{defn}



\begin{lem}[Bracketing complexity of low-rank tensors] \label{lem:metric}
Define the family of rank-$r$ bounded tensors $\tF=\{\tZ\in\mathbb{R}^{d_1\times \cdots \times d_K}\colon \rank(\tZ)\leq r, \ \FnormSize{}{\tZ}\leq 1\}$. Set 
\begin{equation}\label{eq:specification}
L_n\asymp \left({Kr d_{\max} \over n } \right)^{1/(2-\alpha)} + {1\over \rho (\pi, \tN)}\left({Krd_{\max} \over n } \right).
\end{equation}
Then, the following inequality is satisfied.
\begin{equation}\label{eq:L}
{1\over L_n}\int^{\sqrt{L_n^{\alpha/(\alpha+1)}+{L_n\over \rho (\pi, \tN)}}}_{L_n} \sqrt{\tH_{[\ ]}(\varepsilon, \tF, \vnormSize{}{\cdot}) }d\varepsilon \leq n^{1/2}.
\end{equation}
\end{lem}
\begin{proof}
To simplify the notation, we denote $\rho=\rho(\pi, \tN)$. Based on~\citet{mu2014square,9119759}, the covering number for rank-$r$ tensor is
\[
\tH_{[\ ]}(2\varepsilon, \tF, \vnormSize{}{\cdot})\leq \tH(\varepsilon, \tF, \FnormSize{}{\cdot}) \leq CKrd_{\max}\log {d_{\max}\over \varepsilon}.
\]
Inserting the bracketing number into~\eqref{eq:L} gives
\begin{equation}\label{eq:complexity}
g(L)={1\over L}\int^{\sqrt{L^{\alpha/(\alpha+1)}+{\rho^{-1}L}}}_{L}  \sqrt{Krd_{\max}\log\left({d_{\max}\over \varepsilon}\right)}d\varepsilon.
\end{equation}
By the monotonicity of the integrand in~\eqref{eq:complexity}, we bound $g(L)$ by 
\begin{align}\label{eq:g}
g(L)&\leq {\sqrt{Krd_{\max}}\over L}\int_{L}^{\sqrt{L^{\alpha/\alpha+1}+\rho^{-1}L}}\sqrt{\log \left(d_{\max} \over L \right)}d\varepsilon\notag \\
&\leq \sqrt{Krd_{\max}(\log d_{\max} - \log L)}\left({L^{\alpha/(2\alpha+2)}+\sqrt{\rho^{-1}L} \over L }-1\right)\notag \\
&\leq  \sqrt{Krd_{\max}(\log d_{\max})}\left( {1\over L^{(\alpha+2)/(2\alpha+2)}}+{1\over \sqrt{\rho L}}\right),
\end{align}
where the second line follows from $\sqrt{a+b} \leq \sqrt{a}+\sqrt{b}$ for $a,b>0$.
It remains to verify that $g(L_n) \leq n^{1/2}$ for $L_n$ specified in~\eqref{eq:L}. Plugging $L_n$ into the last line of~\eqref{eq:g} gives
\[
g(L_n)\leq\cdots\fixme{Miaoyan}{crude but elegant relaxation is preferred}\leq Cn^{1/2},
\]
where $C>0$ is a constant independent of $r$ and $d$. The proof is therefore complete. 
\end{proof}

\subsection{Proof of Theorem~\ref{thm:estimation}}
\begin{proof}[Proof of Theorem~\ref{thm:estimation}]
Write $n=|\Omega|$. We first prove that 
\begin{equation}\label{eq:total}
{1\over H}\sum_{\pi \in \Pi} \textup{MAE}(\sign \hat Z_\pi, \sign (\Theta-\pi)) \lesssim  t_n^{\alpha/(\alpha+2)}+ {1\over H}+ H t_n, \quad \text{with } t_n={Kd_{\max}r\over n}.
\end{equation}
Theorem~\ref{thm:classification} implies that the  sign estimation accuracy depends on the closeness of $\pi\in \tH$ to the mass points in $\tH$. Therefore, we partition the level set $\pi \in \tH$ based on their closeness to $\tH$. Specifically, let $\tN_H \stackrel{\text{def}}{=}\bigcup_{\pi'\in\tN}\left(\pi'-\frac{1}{H},\pi'+\frac{1}{H}\right)$ denote the set of levels close to the mass points. We expand~\eqref{eq:total} by
\begin{align}\label{eq:twobounds}
&{1\over H}\sum_{\pi \in \Pi} \textup{MAE}(\sign \hat Z_\pi, \sign (\Theta-\pi))\notag \\
=&{1\over H}\sum_{\pi \in \Pi\cap \tN_H} \textup{MAE}(\sign \hat Z_\pi, \sign (\Theta-\pi))+{1\over H}\sum_{\pi \in \Pi\cap \tN_H^c} \textup{MAE}(\sign \hat Z_\pi, \sign (\Theta-\pi)).
\end{align}
By assumption, the first term involves only finite number of summands and thus can be bounded by $2c/H$ where $c>0$ is a constant such that $|\tN|\leq c$.  We bound the second term using the explicit forms of $\rho(\pi, \tN)$ in the sequence $\pi \in\Pi\cap \tN_H^c$. Based on Theorem~\ref{thm:classification}, 
\begin{align}
{1\over H}\sum_{\pi \in \Pi\cap \tN_H^c} \textup{MAE}(\sign \hat \tZ_\pi, \sign (\Theta-\pi)) &\lesssim  {1\over H}\sum_{\pi\in \Pi\cap \tN_H^c} t_n^{\alpha/(\alpha+2)}+{t_n\over H}\sum_{\pi \in \Pi\cap \tN_H^c}{1\over \rho^2(\pi, \tN)}\\
&\leq t_n^{\alpha/(\alpha+2)}+{t_n\over H} \sum_{\pi \in \Pi\cap \tN_H^c} \sum_{\pi' \in \tN}{1\over |\pi-\pi'|^2}\\
&\leq  t_n^{\alpha/(\alpha+2)}+{t_n\over H} \sum_{\pi'\in \tN} \sum_{\pi \in \Pi\cap \tN_H^c}{1\over |\pi-\pi'|^2}\\
&\leq t_n^{\alpha/(\alpha+2)}+ 2cHt_n,
\end{align}
where the last inequality follows from the Lemma~\ref{lem:H}.  Combining the bounds for the two terms in \eqref{eq:twobounds} completes the proof for conclusion~\eqref{eq:total}. 

We then prove
\begin{equation}\label{eq:real}
\textup{MAE}(\hat \Theta, \Theta)\lesssim \left(d_{\max}r \over|\Omega|\right)^{{\alpha \over \alpha+2} \vee {1\over 2}}.
\end{equation}
It follows from the definition of $\hat \Theta$ that, for every $\omega\in[d_1]\times\cdots\times[d_K]$,
\begin{align}
|\hat \Theta(\omega)- \Theta(\omega)|&\leq\frac{1}{H}\sum_{\pi\in\Pi\setminus\tN}|\sign \hat \tZ_\pi(\omega)\neq \sign(\Theta(\omega)-\pi)|+ {1+2c\over 2H},
\end{align}
where we have used the assumption that $|\tN|\leq c$. Taking expectation with respect to $\omega\sim \Pi$ on both sides gives
\begin{align}
\textup{MAE}(\Theta, \hat \Theta)&\lesssim \frac{1}{H}\sum_{\pi\in\Pi\setminus\tN}\textup{MAE}(\sign \hat \tZ_\pi, \sign(\Theta-\pi))+{1\over H}\\
&\lesssim \left({Kd_{\max} r \over |\Omega|}\right)^{\alpha/(\alpha+2)}+{1\over H}+{HKd_{\max} r \over |\Omega|},
\end{align}
where the last line comes from~\eqref{eq:total}. 
\end{proof}


\begin{lem}\label{lem:H}
Fix a $\pi'\in\tN$ and a sequence $\Pi=\{1/H,\ldots,(H-1)/H\}$ with $H\geq 2$. Then, 
\[
\sum_{\pi \in \Pi\cap \tN_H^c}{1\over 
|\pi-\pi'|^2}\leq 2H^2. 
\]
\begin{proof}
Notice that  all points $\pi\in\Pi\cap\tN_H^c$ satisfy 
$|\pi-\pi'|>{1\over H}$ for all $\pi'\in\tN$.
\begin{align}
   \sum_{\pi \in \Pi\cap \tN_H^c}{1\over |\pi-\pi'|^2}&= \sum_{\frac{h}{H}\in\Pi\cap \tN_H^c } {1\over |\frac{h}{H}-\pi'|^2}\\
   &\leq H^2 \sum_{h=1}^{H}{1 \over h^2}\\
 &\leq H^2\left\{ 1+\int_{1}^2{1\over x^2}dx+ \int_{2}^3{1\over x^2}dx+\cdots + \int_{H-1}^H{1\over x^2}dx\right\}\\
&= H^2\left(1+\int^{H}_{1}{1\over x^2}dx\right) \leq 2H^2,
\end{align}
 where the third line uses the monotonicity of ${1\over x^2}$ for $x\geq 1$. 
 \end{proof}
\end{lem}



\bibliographystyle{chicago}

\bibliography{tensor_wang}

\appendix

\end{document}
