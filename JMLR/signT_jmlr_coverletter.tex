\documentclass{article}
\input{jmlr_coverletter_preamble.tex}


\begin{document}
\hfill{\today}

Journal of Machine Learning Research

\bigskip

Dear Editors:

We are writing to submit our manuscript ``\red{Beyond the Signs: Nonparametric Tensor Completion via Sign Series}'' to the \emph{Journal of Machine Learning Research}.

Higher-order tensors have recently attracted increased attention across science and engineering. In this article, We consider the problem of tensor estimation from noisy observations with possibly missing entries. Such data problems arise commonly in recommendation system, social networks, and multiway comparison applications. We develop the model representing the signal tensor of interest based on a series of structured sign tensors. Unlike earlier methods, the sign series representation effectively addresses both low- and high-rank signals, while encompassing many existing tensor models. We provably reduce the tensor estimation problem to a series of structured classification tasks, and  develop a learning reduction machinery to empower existing low-rank tensor algorithms for more challenging high-rank estimation. Excess risk bounds, estimation errors, and sample complexities are established.
We demonstrate the efficacy of our procedure through simulations and real data applications.

We believe our results will be of interest to a very broad readership -- from those interested in theoretical foundations of tensor methods to those in tensor data applications. Our method will help the practitioners efficiently analyze tensor datasets in various areas. Toward this end, the software package has been publicly released at CRAN.

This paper extends our work of \red{Beyond the Signs: Nonparametric Tensor Completion via Sign Series} in Neural Information Processing Systems 34. The JMLR submission covers much deeper depth and extends the published results in a substantive way.  Newly added results summarized below.
\begin{itemize}
\item We have added new examples and propositions that highlight the advantage of using sign-rank in high-dimensional tensor analysis in Section 3.2.
\item We have provided a new insight about the weighted classification loss. Furthermore, we have extended the loss to the generalized weighted classification with a proposition allowing the flexible penalization on the magnitude of deviation in Section 3.3.
\item We have extended the previous theoretical results to unbounded observation and presented the estimation errors according to various types of noise including sub-Gaussian, Bernoulli and Binomial noises in Section 4.2.
\end{itemize}


We suggest the following action editors and referees for our submission.

Action Editors:
\begin{itemize}
    \item \red{Animashree Anandkumar, California Institute of Technology (anima@caltech.edu)}
    \item \red{Genevera Allen, Rice University (gallen@rice.edu)}
    \item \red{First Lastname, Institution (first.lastname@institution.edu)}
    \item \red{First Lastname, Institution (first.lastname@institution.edu)}
    \item \red{First Lastname, Institution (first.lastname@institution.edu)}
\end{itemize}

Reviewers:
\begin{itemize}
    \item \red{First Lastname, Institution (first.lastname@institution.edu)}
    \item \red{First Lastname, Institution (first.lastname@institution.edu)}
    \item \red{First Lastname, Institution (first.lastname@institution.edu)}
    \item \red{First Lastname, Institution (first.lastname@institution.edu)}
    \item \red{First Lastname, Institution (first.lastname@institution.edu)}
\end{itemize}

Our submission has the following keywords: \red{ tensor completion, nonparametric statistics, sign tensor, rate of convergence, sample complexity.}

As the corresponding author, I confirm that none of the co-authors listed below have a conflict of interest with the action editors and referees I suggest above. Further, I confirm that all co-authors below consent to my submission of this manuscript to the \emph{Journal of Machine Learning Research}.

\bigskip

Sincerely,

\medskip

\red{Chanwoo Lee (Department of Statistics, University of Wisconsin-Madison)}\\
\red{Miaoyan Wang (Department of Statistics, University of Wisconsin-Madison)}

\end{document}






































